\documentclass[9pt, aspectratio=169, handout]{beamer}

% +==================================================+
% |   ████████╗██╗  ██╗███████╗███╗   ███╗███████╗   |
% |   ╚══██╔══╝██║  ██║██╔════╝████╗ ████║██╔════╝   |
% |      ██║   ███████║█████╗  ██╔████╔██║█████╗     |
% |      ██║   ██╔══██║██╔══╝  ██║╚██╔╝██║██╔══╝     |
% |      ██║   ██║  ██║███████╗██║ ╚═╝ ██║███████╗   |
% |      ╚═╝   ╚═╝  ╚═╝╚══════╝╚═╝     ╚═╝╚══════╝   |
% +==================================================+

\useinnertheme{rectangles}
\useoutertheme{infolines}

\setbeamertemplate{footline}{%
    \leavevmode%
    \rule{\paperwidth}{0.1pt}\vskip0.2pt%
    \hbox{%
        \begin{beamercolorbox}[wd=.2\paperwidth, ht=2.25ex, dp=1ex, center]{author in head/foot}%
            \usebeamerfont{author in head/foot}\insertshortauthor
        \end{beamercolorbox}%
        \begin{beamercolorbox}[wd=.6\paperwidth, ht=2.25ex, dp=1ex, center]{title in head/foot}%
            \usebeamerfont{title in head/foot}\insertshorttitle
        \end{beamercolorbox}%
        \begin{beamercolorbox}[wd=.2\paperwidth, ht=2.25ex, dp=1ex, center]{date in head/foot}%
            \usebeamerfont{date in head/foot}\insertshortdate\hfill\insertframenumber{} / \inserttotalframenumber\hspace*{2ex}
        \end{beamercolorbox}%
    }%
    \vskip0pt%
}
% hrule under the frametitle, see:
% https://tex.stackexchange.com/questions/343517/beamer-full-width-hrule-below-frame-title
\setbeamertemplate{frametitle}{%
    \usebeamerfont{frametitle}\insertframetitle%
    \vphantom{g}%
    \par%
    \makebox[\linewidth][c]{%
        \rule[0.5\baselineskip]{\paperwidth}{0.4pt}%
    }%
}

% \beamertemplatenavigationsymbolsempty % remove navigation bar

\definecolor{grey80}{HTML}{CCCCCC}
\definecolor{grey90}{HTML}{E6E6E6}
\definecolor{grey95}{HTML}{F2F2F2}

\usepackage[svgnames, x11names]{xcolor}
\usecolortheme[named=DarkSlateGrey]{structure}
\setbeamercolor{background canvas}{bg=white}

\setbeamerfont{frametitle}{series=\bfseries}

\let\oldtiny\tiny
\let\oldscriptsize\scriptsize
\let\oldfootnotesize\footnotesize
\let\oldsmall\small
\let\oldnormalsize\normalsize
\let\oldlarge\large
\let\oldLarge\Large
\let\oldLARGE\LARGE
\let\oldhuge\huge
\let\oldHuge\Huge
\renewcommand*{\footnotesize}{\oldfootnotesize\scriptsize}
\renewcommand*{\normalsize}{\oldnormalsize\oldsmall}
\setbeamerfont{frametitle}{size=\oldnormalsize}

% +=================================+
% |   ██████╗ ██╗██████╗ ███████╗   |
% |   ██╔══██╗██║██╔══██╗██╔════╝   |
% |   ██████╔╝██║██████╔╝███████╗   |
% |   ██╔══██╗██║██╔══██╗╚════██║   |
% |   ██████╔╝██║██████╔╝███████║   |
% |   ╚═════╝ ╚═╝╚═════╝ ╚══════╝   |
% +=================================+

% \usepackage[style=nejm,backend=biber]{biblatex}
% \addbibresource{my.bib}
% \renewcommand*{\bibfont}{\footnotesize}

% +==========================================+
% |   ███╗   ███╗ █████╗ ████████╗██╗  ██╗   |
% |   ████╗ ████║██╔══██╗╚══██╔══╝██║  ██║   |
% |   ██╔████╔██║███████║   ██║   ███████║   |
% |   ██║╚██╔╝██║██╔══██║   ██║   ██╔══██║   |
% |   ██║ ╚═╝ ██║██║  ██║   ██║   ██║  ██║   |
% |   ╚═╝     ╚═╝╚═╝  ╚═╝   ╚═╝   ╚═╝  ╚═╝   |
% +==========================================+

\usepackage{amsmath,amssymb,amsfonts}
\usepackage{cancel}
\usepackage{braket}
\usepackage{bm}
\usepackage{siunitx}
\usepackage{feynmf}
\usepackage{cleveref}
\DeclareGraphicsRule{*}{mps}{*}{}

% +===================================================+
% |   ███████╗██╗ ██████╗ ██╗   ██╗██████╗ ███████╗   |
% |   ██╔════╝██║██╔════╝ ██║   ██║██╔══██╗██╔════╝   |
% |   █████╗  ██║██║  ███╗██║   ██║██████╔╝█████╗     |
% |   ██╔══╝  ██║██║   ██║██║   ██║██╔══██╗██╔══╝     |
% |   ██║     ██║╚██████╔╝╚██████╔╝██║  ██║███████╗   |
% |   ╚═╝     ╚═╝ ╚═════╝  ╚═════╝ ╚═╝  ╚═╝╚══════╝   |
% +===================================================+
\usepackage{subcaption}
\usepackage{makecell}
\usepackage{booktabs}

% +===========================================================+
% |    ██████╗██╗  ██╗██╗███╗   ██╗███████╗███████╗███████╗   |
% |   ██╔════╝██║  ██║██║████╗  ██║██╔════╝██╔════╝██╔════╝   |
% |   ██║     ███████║██║██╔██╗ ██║█████╗  ███████╗█████╗     |
% |   ██║     ██╔══██║██║██║╚██╗██║██╔══╝  ╚════██║██╔══╝     |
% |   ╚██████╗██║  ██║██║██║ ╚████║███████╗███████║███████╗   |
% |    ╚═════╝╚═╝  ╚═╝╚═╝╚═╝  ╚═══╝╚══════╝╚══════╝╚══════╝   |
% +===========================================================+

% \usepackage{xeCJK}
% \setCJKmainfont{PingFang SC}

% +====================================+
% |   ███╗   ███╗██╗███████╗ ██████╗   |
% |   ████╗ ████║██║██╔════╝██╔════╝   |
% |   ██╔████╔██║██║███████╗██║        |
% |   ██║╚██╔╝██║██║╚════██║██║        |
% |   ██║ ╚═╝ ██║██║███████║╚██████╗   |
% |   ╚═╝     ╚═╝╚═╝╚══════╝ ╚═════╝   |
% +====================================+

\usepackage{fontawesome5}
\usepackage{twemojis}
\newcommand{\myhl}[1]{\textcolor{DeepPink}{#1}}
\newcommand{\myhlb}[1]{\textcolor{Blue}{#1}}


\title{MAE 131A Discussion Sections\\ Week 3}
\author{Chuanjin Su}
\institute[UCLA MAE]{Mechanical and Aerospace Engineering Department\\
    University of California, Los Angeles}
\date{Oct 18, 2024}

\begin{document}

\begin{frame}
    \titlepage
\end{frame}

\begin{frame}[allowframebreaks]{Laminar boundary layer problem}
    The boundary layer problem can be set with the mass and the momentum equations:
    \begin{subequations}
        \begin{align}
            \frac{\partial u}{\partial x} + \frac{\partial v}{\partial y} &= 0, \label{eq:mass} \\
            u\frac{\partial u}{\partial x} + v\frac{\partial u}{\partial y} &= \nu\frac{\partial^2 u}{\partial y^2},\label{eq:momentum}
        \end{align}
    \end{subequations}
    with the boundary conditions,
    \begin{subequations}
        \begin{align}
            u(y=0) &=0, \\
            v(y=0) &= 0, \\
            u(y=\delta) &= u_\infty, \\
            \frac{\partial u}{\partial y}\bigg|_{y=\delta} &= 0.
        \end{align}
    \end{subequations}
    By integrating Eq.~\eqref{eq:mass} from $y=0$ to $y=\delta$, since $v(y=0)=0$, it is obtained that,
    \begin{equation}
        v(y=\delta) = -\int_0^\delta \frac{\partial u}{\partial x}dy.
    \end{equation}
    Now, we integrate Eq.~\eqref{eq:momentum} from $y=0$ to $y=\delta$,
    \begin{equation}
        \int_0^\delta u\frac{\partial u}{\partial x}dy + uv\bigg|_0^\delta - \int_0^\delta u\cancelto{-\frac{\partial u}{\partial x}}{\frac{\partial v}{\partial y}}dy = \nu \frac{\partial u}{\partial y}\bigg|_0^\delta.
    \end{equation}
    Therefore,
    \begin{equation}
        \frac{\mathrm{d} }{\mathrm{d} x} \left\lbrace \int_0^\delta (u_\infty - u) u \mathrm{d}y \right\rbrace = \nu \frac{\partial u}{\partial y}\bigg|_{y=0}. \label{eq:velocity_integral}
    \end{equation}

    Now, for the thermal boundary layer, the energy equation is,
    \begin{equation}
        u\frac{\partial T}{\partial x} + v\frac{\partial T}{\partial y} = \alpha\frac{\partial^2 T}{\partial y^2}, \label{eq:energy}
    \end{equation}
    with the boundary conditions,
    \begin{subequations}
        \begin{align}
            T(y=0) &= T_w, \\
            T(y=\delta_T) &= T_\infty, \\
            \left.\frac{\partial T}{\partial y}\right|_{y=\delta_T} &= 0.
        \end{align}
    \end{subequations}
    Integrate Eq.~\eqref{eq:energy} from $y=0$ to $y=\delta_T$, we have,
    \begin{equation}
        \int_0^{\delta_T} u\frac{\partial T}{\partial x}dy + vT\bigg|_0^{\delta_T} - \int_0^{\delta_T} T\cancelto{-\frac{\partial u}{\partial x}}{\frac{\partial v}{\partial y}}dy = \alpha\frac{\partial T}{\partial y}\bigg|_{y=0}^{\delta_T},
    \end{equation}
    resulting in,
    \begin{equation}
        \frac{\mathrm{d} }{\mathrm{d} x} \left\lbrace \int_0^{\delta_T} (T_\infty - T) u \mathrm{d}y \right\rbrace = \alpha \frac{\partial T}{\partial y}\bigg|_{y=0}. \label{eq:temperature_integral}
    \end{equation}

    Eqs.~\cref{eq:velocity_integral,eq:temperature_integral} are the key equations to solve the boundary layer problem.

    \textbf{For example}, if the velocity profile is assumed to be a power-law profile up to the third order, i.e., $u(y)/u_{\infty} = a_1 + a_2 (y/\delta)^2 + a_3(y/\delta)^2 + a_4(y/\delta)^3$, the coefficients can be identified by the boundary conditions such that $a_1=0, a_2 = 3/2, a_3=0, a_4=-1/2$. Thus,
    \begin{equation}
        \frac{u}{u_{\infty}} = \frac{3}{2} \frac{y}{\delta} - \frac{1}{2} \left(\frac{y}{\delta}\right)^3. \label{eq:power_law}
    \end{equation}
    Substituting Eq.~\eqref{eq:power_law} into Eq.~\eqref{eq:velocity_integral}, it is obtained that,
    \begin{equation}
        \frac{\mathrm{d} }{\mathrm{d} x} \left(\frac{39}{280} u_{\infty}^2 \delta \right) = \frac{3}{2}\frac{\nu u_{\infty}}{\delta} \qquad \Rightarrow \delta = \sqrt{\frac{280}{13}} \sqrt{\frac{\nu x}{u_{\infty}}} = 4.64 \mathrm{Re}_x^{-1/2} x
    \end{equation}

    For the thermal boundary layer, if the temperature profile is also taken to be,
    \begin{equation}
        \frac{T - T_w}{T_\infty - T_w} = b_1 + b_2\left(\frac{y}{\delta_T}\right) + b_3\left(\frac{y}{\delta_T}\right)^2 + b_4\left(\frac{y}{\delta_T}\right)^3,
    \end{equation}
    using the boundary conditions, it is identified that $b_1=0, b_2=\frac{3}{2}, b_3=0, b_4=-\frac{1}{2}$. Substituting the temperature profile into Eq.~\eqref{eq:temperature_integral}, it can be shown that $\delta_T/\delta = \mathrm{Pr}^{-1/3}/1.026$.
\end{frame}

\begin{frame}{Problem 1}
    \begin{columns}
        \column{0.8\textwidth}
        The temperature profile in the thermal boundary layer for convection on a horizontal surface for constant wall temperature can be assumed to be a \myhl{second-degree polynomial} function. Obtain the final temperature profile in terms of thermal boundary layer thickness.
    \end{columns}
\end{frame}

\begin{frame}{Problem 1 Solution}
    \textbf{Solution}: Consider the temperature profile $T=Ay^2 + By + C$, the boundary conditions are:
    \begin{equation}
        \left\lbrace
        \begin{aligned}
            T(y=0) &= T_w, \\
            T(y=\delta_T) &= T_\infty, \\
            \left.\frac{\partial T}{\partial y}\right|_{y=\delta_T} &= 0.
        \end{aligned}
        \right. \Rightarrow
        \left\lbrace
        \begin{aligned}
            C &= T_w, \\
            A\delta_T^2 + B\delta_T + T_w &= T_\infty, \\
            2A\delta_T + B &= 0.
        \end{aligned}
        \right.
    \end{equation}
    Thus, $A=\frac{T_w - T_{\infty}}{\delta_T^2}$, $B=\frac{2(T_{\infty}-T_w)}{\delta_T}$, and $C=T_w$, the temperature profile is,
    \begin{equation}
        T(y) = \frac{T_w - T_{\infty}}{\delta_T^2}y^2 + \frac{2(T_{\infty}-T_w)}{\delta_T}y + T_w.
    \end{equation}
\end{frame}

\begin{frame}{Problem 2}
    \begin{columns}
        \column{0.8\textwidth}
        If both the velocity profile and the temperature profile are assumed to be sinusoidal, i.e., $u(y) / u_{\infty} = \sin(\pi y/\delta)$ and $(T(y) - T_w) / (T_{\infty} - T_w) = \sin(\pi y/\delta_T)$, determine the boundary layer thicknesses for the velocity profile. And try to obtain the boundary layer thickness for the temperature profile.
    \end{columns}
\end{frame}

\end{document}
