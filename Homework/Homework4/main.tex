
\documentclass[12pt,letterpaper]{article}


\usepackage[top=1in, 
		    bottom=1in,
		    left=1in,
		    right=1in]{geometry}
\usepackage{setspace}	% makes the \singlespacing, \onehalfspacing, and \doublespacing commands available
% \usepackage[en-US]{datetime2}
% \DTMlangsetup{showdayofmonth=false}
% \usepackage{titlesec}
\usepackage{listings}	% allows for placing programming code to be displayed correctly
\usepackage{siunitx}	% units
\usepackage{amsmath}
\usepackage{amsfonts}
\usepackage{amssymb}
\usepackage{graphicx}
\usepackage{booktabs}
\usepackage{multirow}
\usepackage{pgfplots}
\pgfplotsset{compat=newest}
\usepackage{tikz}
%\usetikzlibrary{shapes.geometric}
% \usepgfplotslibrary{external} 
% \tikzexternalize[prefix=pdfimages/,
% 		        mode=list and make]
\usepackage{caption}
\usepackage[list=true,
		     listformat=simple]{subcaption}
%\usepackage{cleveref}	% this should really go last
% \usepackage[colorlinks,
% 		     linkcolor=black,
% 		     citecolor=black,
% 		     plainpages=false,
% 		     pdfpagelabels]{hyperref}
% \usepackage[all]{hypcap}
\usepackage{cleveref}
% \doublespacing

\pagenumbering{gobble}
\newcommand{\mymder}[2]{\ensuremath{\frac{\mathrm{D}#1}{\mathrm{D}#2}}}
\newcommand{\mypder}[2]{\ensuremath{\frac{\partial #1}{\partial #2}}}
\newcommand{\mypdertwo}[2]{\ensuremath{\frac{\partial^2 #1}{\partial #2^2}}}
\newcommand{\mymdervec}[1]{\ensuremath{mypder{#1}{t} + }}
\newcommand{\myder}[2]{\ensuremath{\frac{d#1}{d#2}}}
\newcommand{\mydiv}[1]{\ensuremath{\nabla \cdot {#1}}}
\newcommand{\myfrac}[2]{\ensuremath{^{#1}\!/_{#2}}}
\newcommand{\myfunc}[2]{\ensuremath{#1 \left( #2 \right)}}
\newcommand{\myparen}[1]{\ensuremath{\left( #1 \right)}}
\newcommand{\mybrack}[1]{\ensuremath{\left[ #1 \right]}}
\newcommand{\mybrace}[1]{\ensuremath{\left\{ #1 \right\}}}
\newcommand{\mysin}[1]{\ensuremath{\myfunc{\mathrm{sin}}{#1}}}
\newcommand{\mycos}[1]{\ensuremath{\myfunc{\mathrm{cos}}{#1}}}
\newcommand{\myexp}[1]{\ensuremath{\myfunc{\mathrm{exp}}{#1}}}
\newcommand{\myint}[4]{\ensuremath{\int_{#1}^{#2} {#3} d {#4}}}


\newcommand{\Rld}{\ensuremath{\mathit{Re}}}
\newcommand{\St}{\ensuremath{\mathit{St}}}
\newcommand{\Prn}{\ensuremath{\mathit{Pr}}}
\newcommand{\Sc}{\ensuremath{\mathit{Sc}}}
\newcommand{\Sh}{\ensuremath{\mathit{Sh}}}
\newcommand{\Nu}{\ensuremath{\mathit{Nu}}}
\newcommand{\Bi}{\ensuremath{\mathit{Bi}}}


\let\textacute\'
\let\textgrave\`


\newcommand{\includetikz}[2]{%
    \tikzsetnextfilename{#2}%
    \input{#1#2.tex}%
}

\begin{document}

\noindent
MECH 131A Homework 4

\noindent
Assigned date: November $3^{\mathrm{th}}$, 2024

\noindent
Due date: November $8^{\mathrm{th}}$, 2024

\subsubsection*{Instructions}
\begin{enumerate}
	\item Indicate the names of your study group members.
	\item Draw a sketch of the problem.
	\item If using a specific solution methodology is specified, please include a copy of the Python/MATLAB script.
	\item If using a specific solution methodology is \textit{not} specified, please include a copy of your computational methodology.
\end{enumerate}


\subsubsection*{Problem Set}
\begin{enumerate}

    \item A \SI{25}{\milli\meter}, high-tension line has an electrical resistance of \SI{1e-4}{\ohm\per\meter} and is transmitting a current of \SI{1000}{\ampere}.
    \begin{enumerate}
        \item Describe how the surface and centerline temperature vary with the Biot number.
            From a thermal perspective, should the line be made out of aluminum or copper?
        \item For an aluminum wire, plot how the surface temperature changes with the Reynolds' number.
    \end{enumerate}
    
    \item Beginning with the equation,
    \begin{equation*}
        \mathit{Nu}_x = \frac{0.339 \mathit{Re}_x^{1/2} \mathit{Pr}^{1/3}}{\left[ 1 + \left( 0.0468 / \mathit{Pr}^{2/3} \right) \right]^{1/4}}, \quad \mathit{Pe}_x > 100,
    \end{equation*}
        show that $\overline{\mathit{Nu}}_L$ for a laminar boundary layer over a flat, isothermal surface is given over the entire range or Prandtl number by this equation:
        \begin{equation*}
            \overline{\mathit{Nu}}_L = \frac{0.677 \mathit{Re}_L^{1/2} \mathit{Pr}^{1/3}}{\left[ 1 + \left( 0.0468 / \mathit{Pr}^{2/3} \right) \right]^{1/4}}
        \end{equation*}

    \item Mercury at \SI{25}{\celsius} at \SI{0.7}{\meter\per\second} over a \SI{4}{\centi\meter} long flat heater at \SI{60}{\celsius}.
        The flow is laminar.
        Find $\overline{h}$, $\overline{\tau}_w$, the local heat transfer coefficient at \SI{0.04}{\meter}, and the momentum boundary layer thickness at \SI{0.04}{\meter}
    
    \item For air flowing above an isothermal plate, plot the ratio of the laminar to turbulent heat transfer coefficient as a function of $\mathit{Re}_x$.
        What does the plot suggest about designing for effective heat transfer?
    
    \item A \SI{21.5}{\kilogram\per\second} of water is dynamically and thermal developed in a \SI{12}{\centi\meter} (inner-diameter) pipe.
        The pipe is held at \SI{90}{\celsius} and $\epsilon / D = 0$.
        Find the heat transfer coefficient, $h$, and the friction factor, $f$, where the bulk temperature of the fluid has reached \SI{50}{\celsius}.
        Compare the answers when using the Dittus-Boelter, Colburn, McAdams, Sieder-Tate, and Gnielinski correlations.
        See \textit{A Heat Transfer Textbook} for a description of the various correlations.
    
    \item Water enters a smooth walled, \SI{7}{\centi\meter} (inner-diameter) pipe at \SI{5}{\celsius} at a bulk velocity of \SI{0.86}{\meter\per\second}.
        The pipe wall is kept at \SI{73}{\celsius} by low pressure steam condensing outside.
        Plot $T_b$ against the position in the pipe until $\left( T_w - T_b \right) / 68 = 0.01$.
        Neglect the entry length, but consider property variations.
    
    \item In the final stages of production, a pharmaceutical is sterilized by heating it from 25 to \SI{75}{\celsius} as it moves at \SI{0.2}{\meter\per\second} through a straight thin-walled stainless steel tube of \SI{12.7}{\milli\meter} diameter.
        A uniform heat flux is maintained by an electric resistance heater wrapped around the outer surface of the tube.
        If the tube is \SI{10}{\meter} long, what is the required heat flux?
        If fluid enters the tube with a fully developed velocity profile and a uniform temperature profile, what is the surface temperature at the tube exit and at a distance of 0.5 m from the entrance?
        Fluid properties may be approximated as $\rho = \SI{1000}{\kilogram\per\cubic\meter}$, $c_p = \SI{4000}{\joule\per\kilogram\per\kelvin}$, $\mu = \SI{2e-3}{\kilogram\per\second\per\meter}$, $k = \SI{0.8}{\watt\per\meter\per\kelvin}$, and $\mathit{Pr} = 10$.
    
    \item A device that recovers heat from high-temperature combustion products involves passing the combustion gas between parallel plates, each of which is maintained at \SI{350}{\kelvin} by water flow on the opposite surface.
        The plate separation is \SI{40}{\milli\meter}, and the gas flow is fully developed.
        The gas may be assumed to have the properties of atmospheric air, and its mean temperature and velocity are \SI{1000}{\kelvin} and \SI{60}{\meter\per\second}, respectively.
        \begin{enumerate}
            \item What is the heat flux at the plate surface?
            \item If a third plate, \SI{20}{\milli\meter} thick, is suspended midway between the original plates, what is the surface heat flux for the original plates?
                Assume the temperature and ow rate of the gas to be unchanged and radiation effects to be negligible.
        \end{enumerate}
\end{enumerate}


\end{document}
