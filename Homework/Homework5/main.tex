
\documentclass[12pt,letterpaper]{article}


\usepackage[top=1in, 
		    bottom=1in,
		    left=1in,
		    right=1in]{geometry}
\usepackage{setspace}	% makes the \singlespacing, \onehalfspacing, and \doublespacing commands available
% \usepackage[en-US]{datetime2}
% \DTMlangsetup{showdayofmonth=false}
% \usepackage{titlesec}
\usepackage{listings}	% allows for placing programming code to be displayed correctly
\usepackage{siunitx}	% units
\usepackage{amsmath}
\usepackage{amsfonts}
\usepackage{amssymb}
\usepackage{graphicx}
\usepackage{booktabs}
\usepackage{multirow}
\usepackage{pgfplots}
\pgfplotsset{compat=newest}
\usepackage{tikz}
%\usetikzlibrary{shapes.geometric}
% \usepgfplotslibrary{external} 
% \tikzexternalize[prefix=pdfimages/,
% 		        mode=list and make]
\usepackage{caption}
\usepackage[list=true,
		     listformat=simple]{subcaption}
%\usepackage{cleveref}	% this should really go last
% \usepackage[colorlinks,
% 		     linkcolor=black,
% 		     citecolor=black,
% 		     plainpages=false,
% 		     pdfpagelabels]{hyperref}
% \usepackage[all]{hypcap}
\usepackage{cleveref}
% \doublespacing

\pagenumbering{gobble}
\newcommand{\mymder}[2]{\ensuremath{\frac{\mathrm{D}#1}{\mathrm{D}#2}}}
\newcommand{\mypder}[2]{\ensuremath{\frac{\partial #1}{\partial #2}}}
\newcommand{\mypdertwo}[2]{\ensuremath{\frac{\partial^2 #1}{\partial #2^2}}}
\newcommand{\mymdervec}[1]{\ensuremath{mypder{#1}{t} + }}
\newcommand{\myder}[2]{\ensuremath{\frac{d#1}{d#2}}}
\newcommand{\mydiv}[1]{\ensuremath{\nabla \cdot {#1}}}
\newcommand{\myfrac}[2]{\ensuremath{^{#1}\!/_{#2}}}
\newcommand{\myfunc}[2]{\ensuremath{#1 \left( #2 \right)}}
\newcommand{\myparen}[1]{\ensuremath{\left( #1 \right)}}
\newcommand{\mybrack}[1]{\ensuremath{\left[ #1 \right]}}
\newcommand{\mybrace}[1]{\ensuremath{\left\{ #1 \right\}}}
\newcommand{\mysin}[1]{\ensuremath{\myfunc{\mathrm{sin}}{#1}}}
\newcommand{\mycos}[1]{\ensuremath{\myfunc{\mathrm{cos}}{#1}}}
\newcommand{\myexp}[1]{\ensuremath{\myfunc{\mathrm{exp}}{#1}}}
\newcommand{\myint}[4]{\ensuremath{\int_{#1}^{#2} {#3} d {#4}}}


\newcommand{\Rld}{\ensuremath{\mathit{Re}}}
\newcommand{\St}{\ensuremath{\mathit{St}}}
\newcommand{\Prn}{\ensuremath{\mathit{Pr}}}
\newcommand{\Sc}{\ensuremath{\mathit{Sc}}}
\newcommand{\Sh}{\ensuremath{\mathit{Sh}}}
\newcommand{\Nu}{\ensuremath{\mathit{Nu}}}
\newcommand{\Bi}{\ensuremath{\mathit{Bi}}}


\let\textacute\'
\let\textgrave\`


\newcommand{\includetikz}[2]{%
    \tikzsetnextfilename{#2}%
    \input{#1#2.tex}%
}

\begin{document}

\noindent
MECH 131A Homework 5

\noindent
Assigned date: November $10^{\mathrm{th}}$, 2024

\noindent
Due date: November $15^{\mathrm{th}}$, 2024

\subsubsection*{Instructions}
\begin{enumerate}
	\item Indicate the names of your study group members.
	\item Draw a sketch of the problem.
	\item If using a specific solution methodology is specified, please include a copy of the Python/MATLAB script.
		Some sort of package that calculates properties, such as CoolProp, will likely be very useful. 
	\item If using a specific solution methodology is \textit{not} specified, please include a copy of your computational methodology.
\end{enumerate}


\subsubsection*{Problem Set}
\begin{enumerate}
    \item The side wall of a house is \SI{10}{\meter} in height.
        The overall heat transfer coefficient between the interior air and exterior surface is \SI{2.5}{\watt\per\square\meter\per\kelvin}.
        On a cold night, still winter night $T_{\mathrm{outside}} = - \SI{30}{\celsius}$ and $T_{\mathrm{inside~air}} = \SI{25}{\celsius}$.
        What is $\overline{h}_{\mathrm{conv}}$ on the exterior wall of the house if $\epsilon = 0.9$?
        Is the external convection laminar or turbulent?
    
    \item A circular grill of diameter \SI{0.25}{\meter} and emissivity of 0.9 is maintained at a constant surface temperature of \SI{130}{\celsius}.
        What electrical power is required when the room air and surroundings are at \SI{24}{\celsius}?
    
    \item The hot and cold inlet temperatures to a concentric tube heat exchanger are $T_{h,i} = \SI{200}{\celsius}$, $T_{c,i} = \SI{100}{\celsius}$, respectively.
        The outlet temperatures are $T_{h,o} = \SI{110}{\celsius}$, $T_{c,o} = \SI{125}{\celsius}$.
        Is the heat exchanger operating in a parallel flow or in a counterflow configuration?
        What is the heat exchanger effectiveness?
        What is the NTU?
        Phase change does not occur in either fluid.

    \item To make lead shot, molten droplets of lead are showered into the top of a tall tower.
        The droplets fall through air and solidify before they reach the bottom of the tower, where they are collected.
        Cool air is introduced at the bottom of the tower and warm air flows out the top.
        For a particular tower, \SI{5000}{\kilogram\per\hour} of \SI{2.8}{\milli\meter} diameter droplets are released at their melting temperature of \SI{600}{\kelvin}.
        The latent heat of solidification is \SI{23.1}{\kilo\joule\per\kilogram}.
        The droplets have a density of \num{6700} droplets per cubic meter in the tower.
        Air enters the bottom at \SI{20}{\celsius} with a mass flow rate of \SI{2400}{\kilogram\per\hour}.
        The tower has an internal diameter of \SI{0.6}{\meter} with adiabatic walls.
        Ignore radiative heat transfer.

        \begin{enumerate}
            \item Derive an expression for the steady-state velocity relative to the velocity of the air for the droplets by balancing the drag force with gravity.
                Do not solve it; you will need it below.
            \item Derive an expression for $UA$.
                Do not solve it; you will need it below.
            \item Determine the air temperature and height at the point where the lead has just finished solidifying.
                In addition, determine the air temperature and additional height if the shot should be \SI{60}{\celsius} when it finally reaches the bottom.
                These are stated as separate questions but will need to be solved simultaneously.
                Use the average air temperature in each section of the flow to calculate air properties.
                You will likely have to iterate to a solution as the properties of the air flow vary as a function of height.
                \begin{enumerate}
                    \item Solve this using the $\epsilon$--NTU methodolgy.
                    \item Solve this using the logarithmic mean temperature difference methodology.
                \end{enumerate}
        \end{enumerate}
    
    \item Saturated steam at 1 atm condenses on the outer surface of a vertical, \SI{100}{\milli\meter} diameter pipe \SI{1}{\meter} long, having a uniform surface temperature of \SI{94}{\celsius}.
        Estimate the total condensation rate and the heat transfer rate to the pipe.
    
    \item  Calculate the critical heat flux on a large horizontal surface for the following fluids at 1 atm: mercury, ethanol, and refrigerant R-134a.
        Compare these results to the critical heat flux for water at 1 atm.
    
    \item Graph an estimate of the boiling curve for isobutane at 1 atm and provide the coordinates of the following points:
        \begin{enumerate}
            \item ONB
            \item CHF
            \item Leidenfrost
        \end{enumerate}
        Briefly discuss why these coordinates are \textit{not} exact.
\end{enumerate}

\end{document}
