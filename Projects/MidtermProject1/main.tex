
\documentclass[12pt,letterpaper]{article}


\usepackage[top=1in, 
		    bottom=1in,
		    left=1in,
		    right=1in]{geometry}
\usepackage{setspace}	% makes the \singlespacing, \onehalfspacing, and \doublespacing commands available
% \usepackage[en-US]{datetime2}
% \DTMlangsetup{showdayofmonth=false}
% \usepackage{titlesec}
\usepackage{listings}	% allows for placing programming code to be displayed correctly
\usepackage{siunitx}	% units
\usepackage{amsmath}
\usepackage{amsfonts}
\usepackage{amssymb}
\usepackage{graphicx}
\usepackage{booktabs}
\usepackage{multirow}
\usepackage{pgfplots}
\pgfplotsset{compat=newest}
\usepackage{tikz}
%\usetikzlibrary{shapes.geometric}
% \usepgfplotslibrary{external} 
% \tikzexternalize[prefix=pdfimages/,
% 		        mode=list and make]
\usepackage{caption}
\usepackage[list=true,
		     listformat=simple]{subcaption}
%\usepackage{cleveref}	% this should really go last
\usepackage[
	colorlinks,
	linkcolor=black,
	citecolor=black,
	plainpages=false,
	pdfpagelabels]{hyperref}
% \usepackage[all]{hypcap}
\usepackage{cleveref}
\input{./formatting}
\newcommand{\mymder}[2]{\ensuremath{\frac{\mathrm{D}#1}{\mathrm{D}#2}}}
\newcommand{\mypder}[2]{\ensuremath{\frac{\partial #1}{\partial #2}}}
\newcommand{\mypdertwo}[2]{\ensuremath{\frac{\partial^2 #1}{\partial #2^2}}}
\newcommand{\mymdervec}[1]{\ensuremath{mypder{#1}{t} + }}
\newcommand{\myder}[2]{\ensuremath{\frac{d#1}{d#2}}}
\newcommand{\mydiv}[1]{\ensuremath{\nabla \cdot {#1}}}
\newcommand{\myfrac}[2]{\ensuremath{^{#1}\!/_{#2}}}
\newcommand{\myfunc}[2]{\ensuremath{#1 \left( #2 \right)}}
\newcommand{\myparen}[1]{\ensuremath{\left( #1 \right)}}
\newcommand{\mybrack}[1]{\ensuremath{\left[ #1 \right]}}
\newcommand{\mybrace}[1]{\ensuremath{\left\{ #1 \right\}}}
\newcommand{\mysin}[1]{\ensuremath{\myfunc{\mathrm{sin}}{#1}}}
\newcommand{\mycos}[1]{\ensuremath{\myfunc{\mathrm{cos}}{#1}}}
\newcommand{\myexp}[1]{\ensuremath{\myfunc{\mathrm{exp}}{#1}}}
\newcommand{\myint}[4]{\ensuremath{\int_{#1}^{#2} {#3} d {#4}}}


\newcommand{\Rld}{\ensuremath{\mathit{Re}}}
\newcommand{\St}{\ensuremath{\mathit{St}}}
\newcommand{\Prn}{\ensuremath{\mathit{Pr}}}
\newcommand{\Sc}{\ensuremath{\mathit{Sc}}}
\newcommand{\Sh}{\ensuremath{\mathit{Sh}}}
\newcommand{\Nu}{\ensuremath{\mathit{Nu}}}
\newcommand{\Bi}{\ensuremath{\mathit{Bi}}}


\let\textacute\'
\let\textgrave\`


\newcommand{\includetikz}[2]{%
    \tikzsetnextfilename{#2}%
    \input{#1#2.tex}%
}

\begin{document}

\noindent
MECH 131A Midterm Project

\noindent
Assigned date: November $3^{\mathrm{th}}$, 2024

\noindent
Due date: November $17^{\mathrm{th}}$, 2024

\subsubsection*{Instructions}
\begin{enumerate}
	\item This is a project for up to two people.
	\item If using a specific solution methodology is specified, please include a copy of the Python/MATLAB script.
	\item If using a specific solution methodology is \textit{not} specified, please include a copy of your computational methodology.
\end{enumerate}

\subsubsection*{Radiative Heating of Water}

This project further explores the radiative heating of water assigned in homework 2.
So far for this problem, you have added a resistance network between the water and the plate, as well as modeling the effect of radiative heat transfer from the plate.
Now, you will also add in the convection heat transfer from a radiating surface as well as convective heat transfer from the water.
This will continue to be a 1-D problem, where the temperature variation in the direction of the flow of the water is neglected.

\begin{enumerate}
    \item Draw a sketch of the set up starting from the image in the text book.
        Add in the radiating surface and draw all the heat transfers in the system.
        You can assume the net incident solar radiation transfer is the \SI{800}{\watt\per\square\meter}.
    \item Set up the description of the energy balance of the radiating surface accounting for the heat transfer to and from the surface.
        As in the second homework, assume it has the same emissivity as the plate, and has a net heat transfer with the plate of \SI{800}{\watt\per\square\meter}.
        Further, use the average heat transfer coefficient to describe the convective heat transfer from the radiative surface.
        Two emissivities were given in homework 2 problem 1, which one would you pick if designing this system?
        Use that emissivity for this project.
    \item You now have a set of two energy equations; one for the radiating surface and one for the plate.
    	Adding to the the methodology outlined in homework 2, question 1, part b, create three contour plots of the net heat transfer to the water as a function of the Reynolds number over the radiating surface and the Reynolds number in the pipe.
	\begin{itemize}
            \item You can assume \SI{800}{\watt\per\square\meter} is constant for each case.
            \item Use the fin equation that accounts for changing radiative heat transfer along the fin.
            \item Each contour plot should have a different ratio of convective resistance between the water and pipe and the sum of the conduction resistance of the pipe and the contact resistance between the pipe and plate; use ratios of 0.1, 1, and 10.
                This accounts to varying the relative sizes of the terms within the Biot number describing the heat transfer from the water to the plate.
            \item Assume the ambient air has a temperature of \SI{295}{K}.
        \end{itemize}
\end{enumerate}


\end{document}
