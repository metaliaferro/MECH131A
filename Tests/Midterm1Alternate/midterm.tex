
\documentclass[12pt,letterpaper]{article}


\usepackage[top=1in, 
		    bottom=1in,
		    left=1in,
		    right=1in]{geometry}
\usepackage{setspace}	% makes the \singlespacing, \onehalfspacing, and \doublespacing commands available
% \usepackage[en-US]{datetime2}
% \DTMlangsetup{showdayofmonth=false}
% \usepackage{titlesec}
\usepackage{listings}	% allows for placing programming code to be displayed correctly
\usepackage{siunitx}	% units
\usepackage{amsmath}
\usepackage{amsfonts}
\usepackage{amssymb}
\usepackage{graphicx}
\usepackage{booktabs}
\usepackage{multirow}
\usepackage{pgfplots}
\pgfplotsset{compat=newest}
\usepackage{tikz}
%\usetikzlibrary{shapes.geometric}
% \usepgfplotslibrary{external} 
% \tikzexternalize[prefix=pdfimages/,
% 		        mode=list and make]
\usepackage{caption}
\usepackage[list=true,
		     listformat=simple]{subcaption}
%\usepackage{cleveref}	% this should really go last
\usepackage[
	colorlinks,
	linkcolor=black,
	citecolor=black,
	plainpages=false,
	pdfpagelabels]{hyperref}
% \usepackage[all]{hypcap}
\usepackage{cleveref}
\input{./formatting}
\newcommand{\mymder}[2]{\ensuremath{\frac{\mathrm{D}#1}{\mathrm{D}#2}}}
\newcommand{\mypder}[2]{\ensuremath{\frac{\partial #1}{\partial #2}}}
\newcommand{\mypdertwo}[2]{\ensuremath{\frac{\partial^2 #1}{\partial #2^2}}}
\newcommand{\mymdervec}[1]{\ensuremath{mypder{#1}{t} + }}
\newcommand{\myder}[2]{\ensuremath{\frac{d#1}{d#2}}}
\newcommand{\mydiv}[1]{\ensuremath{\nabla \cdot {#1}}}
\newcommand{\myfrac}[2]{\ensuremath{^{#1}\!/_{#2}}}
\newcommand{\myfunc}[2]{\ensuremath{#1 \left( #2 \right)}}
\newcommand{\myparen}[1]{\ensuremath{\left( #1 \right)}}
\newcommand{\mybrack}[1]{\ensuremath{\left[ #1 \right]}}
\newcommand{\mybrace}[1]{\ensuremath{\left\{ #1 \right\}}}
\newcommand{\mysin}[1]{\ensuremath{\myfunc{\mathrm{sin}}{#1}}}
\newcommand{\mycos}[1]{\ensuremath{\myfunc{\mathrm{cos}}{#1}}}
\newcommand{\myexp}[1]{\ensuremath{\myfunc{\mathrm{exp}}{#1}}}
\newcommand{\myint}[4]{\ensuremath{\int_{#1}^{#2} {#3} d {#4}}}


\newcommand{\Rld}{\ensuremath{\mathit{Re}}}
\newcommand{\St}{\ensuremath{\mathit{St}}}
\newcommand{\Prn}{\ensuremath{\mathit{Pr}}}
\newcommand{\Sc}{\ensuremath{\mathit{Sc}}}
\newcommand{\Sh}{\ensuremath{\mathit{Sh}}}
\newcommand{\Nu}{\ensuremath{\mathit{Nu}}}
\newcommand{\Bi}{\ensuremath{\mathit{Bi}}}


\let\textacute\'
\let\textgrave\`


\newcommand{\includetikz}[2]{%
    \tikzsetnextfilename{#2}%
    \input{#1#2.tex}%
}

\begin{document}

\noindent
NAME:

\noindent
MECH 131A Midterm

\noindent
Date: November $4^{\mathrm{th}}$, 2024

\subsection*{Midterm Questions}

\begin{enumerate}

    %% Non-dimensional numbers
    \item Give a brief description of the following non-dimensional numbers.
        Describe what they mean and what they indicate for a physical situation.

    \begin{enumerate}
        \item $\mathit{Fo} = \frac{\alpha t}{L^2}$ \\ \\ \\ \\ \\ \\ \\ \\
        \item $\Prn = \frac{\nu}{\alpha}$ \\ \\ \\ \\ \\ \\ \\ \\
        \item $\mathit{Nu} = \frac{h L}{k}$ \\ \\ \\ \\ \\ \\ \\
    \end{enumerate}
    \newpage

    \item Two walls of widths $L_1$ and $L_2$ are separated by a small cavity held at $T_c$ by condensation.
        The walls are both exposed to the same ambient air and are made of different materials.
        \begin{enumerate}
            \item Sketch the temperature profile assuming they both have the same heat transfer coefficient describing the convective heat transfer to the ambient air.
            \item Define a Biot number for each wall.
            \item Which wall has the larger heat transfer?
                The one with the larger thermal conductivity or the one with the smaller thermal conductivity?
        \end{enumerate}
    \newpage

    %% Semi-infinite conduction
    \item A thick concrete slab has been baking in the sun and has a uniform temperature of \SI{150}{\celsius}.
        Suddenly a rainstorm deposits $\delta$ thickness of water onto concrete slab.
        As an initial estimate you can assume the water cools down and then boils to estimate the length of time to get for the concrete to dry.
        
    \begin{enumerate}
        \item Sketch how the heat transfer at the surface changes with time.
        \item A basic energy balance on the liquid shows that the total energy to get from \SI{150}{\celsius} to \SI{100}{\celsius} is

        \begin{equation*}
            \Delta E_1 = m c_p (T_1 - T_2),
        \end{equation*}

        Another basic energy balance on the liquid after it cools to \SI{100}{\celsius} and then boils off is,

        \begin{equation*}
            \Delta E_2 = m h_{fg}.
        \end{equation*}
        
        Approximately how long will it take for the concrete to be dry?
    \end{enumerate}
    \newpage

    % %% Forced convection
    % \item Fluid velocities can be measured with a thermistor.
    %     A thermistor with a diameter of \SI{1}{\milli\meter} is connected to an electronic circuit that maintains a constant resistance, $R_w$, by adjusting the voltage, $V_w$.
    %     The electrical power dissipated in the wire, $V_w^2 / R_w$, is convected away by the surface of the wire.
    %     Show that
    %     \begin{equation*}
    %         V_w^2 = \left( T_w - T_\infty \right) \left( A + B u^{1/2} \right)
    %     \end{equation*}
    %     and provide estimates of $A$ and $B$.
    %     Note that the kinematic viscosity of air, $\nu$, is approximately \SI{15e-6}{\square\meter\per\second}.
    %     Start by estimating the approximate size of $\Rld_D$.
    %     Assume the thermistor is roughly spherical.
    % \newpage
    
    \item A tall cylinder containing is half filled with cryogenic liquid.
        Assuming the heat coefficient from the ambient to the tank, from the tank to the liquid, and the tank to the gas are all different, give the solution for the temperature profile of the tank wall.
        The gas is at $T_g$, the liquid at $T_l$, and ambient at $T_\infty$.
    \newpage

   \item An array of parallel rectangular fins are used to increase heat transfer from a heated surface to the ambient air flow with separation between the fins of $w$.
       About how long should the fins be in the direction of the flow before the fin performance starts degrading?
       Note, the air is moving parallel with the base of the fins, not towards the base of the fins.
       You can assume the effective free stream velocity of the flow does not change in the streamwise direction for this problem.
       Give your answer in terms of $w$ and $\mathit{Re}$ for both fully laminar flow and flow that is tripped into turbulence at the leading edge so it is turbulent for the whole flow length.
\end{enumerate}

\end{document}
